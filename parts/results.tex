\documentclass[../main.tex]{subfiles}

\begin{document}

\subsection{Determinación de la masa del cuerpo}

En la tabla \ref{ref:table1}, el número de posiciones (u), indica la distancia del jinetillo al punto de equilibrio de la balanza,
 cada posición tiene una longitud de 2 cm o 0,02 m en el SI, medida con una regla de incertidumbre 0,0005 m.
Para determinar la masa de cada pesa, se emplea la segunda condición de equilibrio mecánico, que establece que el momento 
resultante debe ser nulo. Por tanto:

\begin{equation}
    \sum \vec{\tau}^{\: r} = 0
\end{equation}

Como las fuerzas que equilibran el contrapeso se encuentran en la misma dirección, se trabaja el vector \textit{torque} 
de forma unidimensional.

\begin{equation}
    \sum \tau^{\: r} = 0
\end{equation}
\[(torque\ de\ la\ masa)-(suma\ de\ torques\ de\ los\ jinetillos)=0\]

\subsubsection*{Masa de Plomo ($m_1$):}
\[\left(F_P)(d_P\right)-\left[\left(F_{J1}\right)\left(d_{J1}\right)+\left(F_{J2}\right)\left(d_{J2}\right)+\left(F_{J3}\right)\left(d_{J3}\right)+\left(F_{J4}\right)\left(d_{J4}\right)\right]=0\]
\[\left(m_1)(g)(d_P\right)-\left[\left(m_{J1}\right)\left(g\right)\left(d_{J1}\right)+\left(m_{J2}\right)\left(g\right)\left(d_{J2}\right)+\left(m_{J3}\right)\left(g\right)\left(d_{J3}\right)+\left(m_{J4}\right)\left(g\right)\left(d_{J4}\right)\right]=0\]

Donde “$F_p$” representa el peso de la masa de plomo, “$g$” es la aceleración de la gravedad y “$d$” 
con subíndice, es la distancia en posiciones (u) del objeto correspondiente al punto de equilibrio.

\[g\left\{\left(m_1\right)\left(d_P\right)-\left[\left(m_{J1}\right)\left(d_{J1}\right)+\left(m_{J2}\right)\left(d_{J2}\right)+\left(m_{J3}\right)\left(d_{J3}\right)+\left(m_{J4}\right)\left(d_{J4}\right)\right]\right\}=0\]

La aceleración de la gravedad se anula, y despejando la masa de plomo, se obtiene la siguiente expresión:

\[m_1=\frac{\left(m_{J1}\right)\left(d_{J1}\right)+\left(m_{J2}\right)\left(d_{J2}\right)+\left(m_{J3}\right)\left(d_{J3}\right)+\left(m_{J4}\right)\left(d_{J4}\right)}{d_P}\]

Reemplazando los valores:

\[m_1=\frac{\left(0,011\right)\left(8\right)+\left(0,101\right)\left(1\right)+\left(0,101\right)\left(5\right)+\left(0,196\right)\left(2\right)}{10}\]

\[m_1=0,1086\ kg\]

Tomando en cuenta el valor de las masas con el menor número de decimales por tratarse de una suma:

\[m_1\approx0,109\ kg\]

Para hallar las incertidumbres respectivas se realiza el siguiente procedimiento:

\begin{equation} \label{calc1}
    \delta m = \frac{1,09 \times 0,02}{10\times 0,02} 
    \left[ \frac{ \sum_{i=1}^n [m_i \Delta d_i + d_i \Delta m_i]}{1,09\times 0,02}
    +\frac{0,0005}{10\times 0,02} \right]
\end{equation}

Donde “$d_i$” es igual a una “u” por 0,02 metros.

Entonces, reemplazando:

\begin{equation*}
    \Delta m = \frac{1,09 \times 0,02}{10\times 0,02} 
    \left[ \frac{ \sum_{i=1}^4 [m_i \Delta d_i + d_i \Delta m_i]}{1,09\times 0,02}
    +\frac{0,0005}{10\times 0,02} \right]
\end{equation*}

\[\Delta m_1 = 2,895\times10^{-3}\]
\[\Delta m_1\approx 0,003 kg\]

La medida de la masa de plomo ($m_1$), con su respectiva incertidumbre es:

\[m_1\pm\Delta m_1=0,109\pm0,003 kg\]

La incertidumbre relativa es:

\[\varepsilon_r=\frac{0,003}{0,109}\approx0,03=3\%\]


\end{document}