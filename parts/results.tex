\documentclass[../main.tex]{subfiles}

\begin{document}

\subsection{Determinación de la masa del cuerpo}

En la tabla \ref{ref:table1}, el número de posiciones (u), indica la distancia del jinetillo al punto de equilibrio de la balanza,
 cada posición tiene una longitud de 2 cm o 0,02 m en el SI, medida con una regla de incertidumbre 0,0005 m.
Para determinar la masa de cada pesa, se emplea la segunda condición de equilibrio mecánico, que establece que el momento 
resultante debe ser nulo. Por tanto:

\begin{equation}
    \sum \vec{\tau}^{\: r} = 0
\end{equation}

Como las fuerzas que equilibran el contrapeso se encuentran en la misma dirección, se trabaja el vector \textit{torque} 
de forma unidimensional.

\begin{equation}
    \sum \tau^{\: r} = 0
\end{equation}
\[(torque\ de\ la\ masa)-(suma\ de\ torques\ de\ los\ jinetillos)=0\]

\subsubsection*{Masa de Plomo ($m_1$):}
\[\left(F_P)(d_P\right)-\left[\left(F_{J1}\right)\left(d_{J1}\right)+\left(F_{J2}\right)\left(d_{J2}\right)+\left(F_{J3}\right)\left(d_{J3}\right)+\left(F_{J4}\right)\left(d_{J4}\right)\right]=0\]
\[\left(m_1)(g)(d_P\right)-\left[\left(m_{J1}\right)\left(g\right)\left(d_{J1}\right)+\left(m_{J2}\right)\left(g\right)\left(d_{J2}\right)+\left(m_{J3}\right)\left(g\right)\left(d_{J3}\right)+\left(m_{J4}\right)\left(g\right)\left(d_{J4}\right)\right]=0\]

Donde “$F_p$” representa el peso de la masa de plomo, “$g$” es la aceleración de la gravedad y “$d$” 
con subíndice, es la distancia en posiciones (u) del objeto correspondiente al punto de equilibrio.

\[g [ \left(m_1\right)\left(d_P\right)-\left[\left(m_{J1}\right)\left(d_{J1}\right)+\left(m_{J2}\right)\left(d_{J2}\right)+\left(m_{J3}\right)\left(d_{J3}\right)+\left(m_{J4}\right)\left(d_{J4}\right)\right]]=0\]

La aceleración de la gravedad se anula, y despejando la masa de plomo, se obtiene la siguiente expresión:

\[m_1=\frac{\left(m_{J1}\right)\left(d_{J1}\right)+\left(m_{J2}\right)\left(d_{J2}\right)+\left(m_{J3}\right)\left(d_{J3}\right)+\left(m_{J4}\right)\left(d_{J4}\right)}{d_P}\]

Reemplazando los valores:

\[m_1=\frac{\left(0,011\right)\left(8\right)+\left(0,101\right)\left(1\right)+\left(0,101\right)\left(5\right)+\left(0,196\right)\left(2\right)}{10}\]

\[m_1=0,1086\ kg\]

Tomando en cuenta el valor de las masas con el menor número de decimales por tratarse de una suma:

\[m_1\approx0,109\ kg\]

Para hallar las incertidumbres respectivas se realiza el siguiente procedimiento:

\begin{equation} \label{calc1}
    \delta m = \frac{1,09 \times 0,02}{10\times 0,02} 
    \left[ \frac{ \sum_{i=1}^n [m_i \Delta d_i + d_i \Delta m_i]}{1,09\times 0,02}
    +\frac{0,0005}{10\times 0,02} \right]
\end{equation}

Donde “$d_i$” es igual a una “u” por 0,02 metros.

Entonces, reemplazando:

\begin{equation*}
    \Delta m = \frac{1,09 \times 0,02}{10\times 0,02} 
    \left[ \frac{ \sum_{i=1}^4 [m_i \Delta d_i + d_i \Delta m_i]}{1,09\times 0,02}
    +\frac{0,0005}{10\times 0,02} \right]
\end{equation*}

\[\Delta m_1 = 2,895\times10^{-3}\]
\[\Delta m_1\approx 0,003 kg\]

La medida de la masa de plomo ($m_1$), con su respectiva incertidumbre es:

\[m_1\pm\Delta m_1=0,109\pm0,003 kg\]

La incertidumbre relativa es:

\[\varepsilon_r=\frac{0,003}{0,109}\approx0,03=3\%\]

Una incertidumbre muy baja con respecto a la medida, por lo cual, 
posee una alta precisión. Asimismo, al comparar el resultado 
calculado con la medición de la masa de plomo en la balanza 
digital (véase la tabla \ref{ref:table3}), se observa que la medida es 
idéntica en magnitud, pero respecto a la incertidumbre es
menos precisa que la realizada con la balanza digital
(30 veces mayor), no obstante, tratándose de una balanza 
experimental sujeta a la percepción humana al momento de medir,
se puede concluir que es muy eficiente para medir masas.

\subsubsection*{Masa de Bronce ($m_2$):}

\[(F_B)(d_B )-[(F_{J2} )(d_{J2} )+(F_{J5} )(d_{J5} )]=0\]
\[(m_2)(g)(d_B )-[(m_{J2} )(g)(d_{J2} )+(m_{J5} )(g)(d_{J5} )]=0\]

Donde “$F_B$” representa el peso de la masa de bronce, “$g$” es 
la aceleración de la gravedad y “$d$” con subíndice, es la
distancia en posiciones (u) del objeto correspondiente al
punto de equilibrio.

\[g[(m_2)(d_B )-[(m_{J2} )(d_{J2} )+(m_{J5} )(d_{J5} )]]=0\]

La aceleración de la gravedad se anula, y despejando la masa
de bronce,
se obtiene la siguiente expresión:

\[m_2=\frac{((m_{J2} )(d_{J2} )+(m_{J5} )(d_{J5} ))}{d_B} \]

Reemplazando los valores:

\[m_2=\frac{((0,101)(3)+(0,193)(9))}{10}\]

\[m_2=0,204 kg\]

Al igual que con la masa de plomo, para hallar la incertidumbre 
se emplea la ecuación (\ref{calc1}):

\begin{equation*}
    \Delta m_1= \frac{(2,04\times 0,02)}{(10\times 0,02)} 
    \left[\frac{((m_{J2} )(\Delta d_{J2} )+(d_{J2} )(\Delta m_{J2} )
    +(m_{J5} )(\Delta d_{J5} )+
    (d_{J5} )(\Delta m_{J5} ))}{(2,04\times0,02)}
    + \frac{0,0005}{(10\times0,02)}\right]
\end{equation*}

\[\Delta m_2=2,445\times10^{-3}\]

\[\Delta m_2\approx0,002 kg\]

La medida de la masa de bronce ($m_2$), con su respectiva 
incertidumbre es:
\[m_2\pm\Delta m_2=0,204\pm0,002 kg\]
La incertidumbre relativa es:
\[\epsilon_r=\frac{0,002}{0,204}\approx0,01=1\%\]

Nótese que el error relativo de la medida de la masa de 
bronce con la balanza experimental, es 3 veces menor que la 
medida de la masa de plomo con la misma balanza. Esto sucede 
debido a que, la cantidad de jinetillos empleados es menor, lo 
cuál implica, una menor propagación del error.

\subsubsection*{Masa de Tecnopor ($m_3$):}

\[(F_T)(d_T )-[(F_{J1} )(d_{J1} )+(F_{J2} )(d_{J2} )]=0\]
\[(m_3)(g)(d_T )-[(m_{J1} )(g)(d_{J1} )+(m_{J2} )(g)(d_{J2} )]=0\]

Donde “$F_T$” representa el peso de la masa de Tecnopor, “$g$”
es la aceleración de la gravedad y “$d$” con subíndice,
es la distancia en posiciones (u) del objeto correspondiente al 
punto de equilibrio.

\[g[(m_3)(d_T )-[(m_{J1} )(d_{J1} )+(m_{J2} )(d_{J2} )]]=0\]

La aceleración de la gravedad se anula, y despejando la masa de 
Tecnopor, 
se obtiene la siguiente expresión:

\begin{equation*}
    m_3=\frac{(m_{J1} )(d_{J1} )+(m_{J2} )(d_{J2} )}{d_T} 
\end{equation*}

Reemplazando los valores:

\[m_3=\frac{(0,011)(4)+(0,101)(1)}{10}\]
\[m_3=0,0145 kg\]
\[m_3\approx0,015 kg\]

Empleando la ecuación (\ref{calc1}) para determinar la incertidumbre:
\begin{equation*}
    \Delta m_3=\frac{(0,15\times0,02)}{(10\times0,02)}\times
    \left[\frac{(m_{J1} )(\Delta d_{J1} )+(d_{J1} )(\Delta m_{J1} )+
    (m_{J2} )(\Delta d_{J2} )+(d_{J2} )(\Delta m_{J2} )}{(0,15\times0,02)}
    +\frac{0,0005}{10\times0,02}\right]
\end{equation*}

\[\Delta m_3=8,175\times10^{-4}\]
\[\Delta m_3\approx0,001 kg\]
La medida de la masa de Tecnopor ($m_3$), 
con su respectiva incertidumbre es:
\[m_3\pm\Delta m_3=0,015\pm0,001 kg\]
La incertidumbre relativa es:
\[\epsilon_r=\frac{0,001}{0,015}\approx0,07=7\%\]

La incertidumbre relativa en este caso es mayor que la del plomo 
y la del bronce juntos. Esto significa que, mientras más 
pequeña sea la magnitud como en el caso del 
Tecnopor (0,015 kg), menos precisa será la medida. 
A pesar de que las masas de los jinetillos eran menores que en 
casos anteriores, lo cual debería haber reducido la incertidumbre, 
su efecto no fue tan significativo como se esperaba.

\subsection{Determinación del Empuje y el volumen de 3 cuerpos por el Principio de Arquímedes}

Se conoce por la ecuación (\ref{empuje_eq}) que:

\begin{align} 
    \Vec{E} = - \rho_{fluido} \cdot V_{objeto} \cdot \vec{g} \nonumber
\end{align}

\[ |\vec{E}|=|- \rho_{fluido}\cdot V_{objeto} \cdot \vec{g} | \]
\begin{equation} \label{calc_21} 
    E= \rho_{fluido}\cdot V_{objeto} \cdot g 
\end{equation}

El volumen del objeto (masas de bronce, plomo y Tecnopor), se pueden expresar en función de su densidad y masa:

\begin{equation} \label{calc_22}
    V_{objeto}=\frac{m_{objeto}}{\rho_{objeto}}
\end{equation} 

Reemplazando \ref{calc_22} en \ref{calc_21}:
\[E=\rho_{fluido}\cdot\left( \frac{m_{objeto}}{\rho_{objeto}} \right)\cdot g\]

Donde “$\rho$” representa la densidad, “V” el volumen y “E”, el módulo de la fuerza de empuje.

Luego, para determinar la fuerza de empuje, es preciso emplear la segunda condición de equilibrio.

Sabiendo que la relación entre las fuerzas es la siguiente:

\[\vec{T} = \vec{P} - \vec{E}\]
\[\vec{E} = \vec{P} - \vec{T}\]

Donde “$\vec{T}$” representa la fuerza de tensión de la cuerda, “$\vec{P}$” el peso del objeto,
y “$\vec{E}$” el empuje que ejerce el fluido sobre el cuerpo.

Dado que las fuerzas actúan en una sola dirección (vertical). Entonces:

\[|\vec{E}|=|\vec{P}|-|\vec{T}|\]

\begin{align} \label{calc_23}
    E=P-T
\end{align}
Reemplazando (\ref{calc_23}) en (\ref{equilibrio_2}):

\[\tau_E-\tau_P+\tau_T=0\]
\[\tau_E=\tau_P-\tau_T \]

Donde “$\vec{\tau}_E$”, “$\vec{\tau}_P$” y “$\vec{\tau}_T$” representan los módulos de los torques ejercidos 
por las fuerzas de empuje, peso y tensión respectivamente.

Pero la tensión es en realidad, el torque que ejercen los jinetillos para equilibrar el contrapeso. Por tanto:

\begin{equation} \label{calc_24}
    E\cdot d=m\cdot g\cdot d - g \cdot \sum_{i=1}^n (m_i\cdot d_i )
\end{equation}

Donde “d” es la distancia en metros de la pesa al punto de
equilibrio de la balanza de Arquímedes, mientras que la 
sumatoria representa, la suma del producto de las masas por 
las posiciones de cada jinetillo utilizado para equilibrar el 
contrapeso en cada uno de los tres casos.

Reemplazando (\ref{calc_21}) en (\ref{calc_24}):

\[\rho_{fluido}\cdot V_{objeto}\rho g=m\cdot g\cdot d-g\cdot\sum_{i=1}^n (m_i\cdot d_i ) \]
\[\rho_{fluido}\cdot V_{objeto}=m\cdot d-\sum_{i=1}^n (m_i\cdot d_i ) \]

Luego, para determinar el volumen de cada uno de los tres objetos (dado que se encuentran completamente sumergidos):

\begin{equation} \label{calc_25}
    V_{objeto}=\frac{(m\cdot d -\sum_{i=1}^n (m_i\cdot d_i ) )}{\rho_{fluido}} 
\end{equation}

\subsection{Volumen de la masa de plomo ($V_1$):}

\[V_1=\frac{m_1\cdot d_1- (m_{J_3} \cdot d_{J_3}+m_{J_5} \cdot d_{J_5} )}{\rho_{agua}} \]
Reemplazando valores:
\[V_1=\frac{(0,109)(0,20)-[(0,101)(0,08)+(0,193)(0,06)]}{998}\]
\[V_1\approx2,14\times10^{-6}\,  m^3 \]

Expresado en $cm^3$:

\[V_1=2,14cm^3\]

La incertidumbre puede expresarse de la siguiente manera:

\begin{align} \nonumber
    \Delta V=m\cdot\Delta d+d\cdot\Delta m+\sum_{i=1}^n (m_i \cdot\Delta d_i+d_i \cdot\Delta m_i ) 
\end{align}

Dado que, la incertidumbre de la distancia “$\Delta d$”, se debe exclusivamente a la precisión de la
regla utilizada para realizar todas las mediciones, entonces:

\begin{equation} \label{calc_26}
    \Delta V=d\cdot\Delta m+\Delta\cdot m_i \cdot \sum_{i=1}^n (d_i ) +\Delta d\left[\sum_{k=1}^s(m_k ) +m\right]
\end{equation}

En esta ecuación, se considera la densidad del agua como un valor teórico idealizado, ya que el valor utilizado es ampliamente aceptado en la literatura científica y se ha obtenido mediante métodos de medición más precisos y sofisticados que los utilizados en el experimento actual.

De este modo, la incertidumbre del volumen de la pesa de plomo es:

\[Delta V_1=(0,20)(0,003)+(0,001)(0,08+0,06)+(0,0005)(0,101+0,193+0,109)\]

\[\Delta V_1\approx9,42\times 10^{-4} \, m^3\]

Expresado en cm3:
\[\Delta V_1\approx942 \, cm^3\]

\end{document}
