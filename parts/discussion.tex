\documentclass[../main.tex]{subfiles}

\begin{document}
En los resultados del primer experimento, es notable la diferencia del error relativo del objeto más ligero con los otros dos. Resulta importante notar que, aún sin que varíe notablemente el error en sus valores posibles, el mismo es más significativo respecto al valor principal conforme este último es más pequeño. Asimismo, es razonable afirmar que resulta deseable minimizar el número de jinetillos empleados en las mediciones, pues aquello reduce el error resultante.

En los resultados del segundo experimento, resalta especialmente la forma en que escala el error obtenido en cada caso. Resulta razonable atribuir la magnitud relativa de los errores a la acumulación de error en el procedimiento de cálculo, especialmente en la manipulación de la información obtenida por equivalencia de torque.

En ambos experimentos, es notable que no fue estandarizada la perspectiva para observar la balanza de brazos, lo que puede haber introducido errores por paralaje.
\end{document}
